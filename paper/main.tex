\documentclass[11pt]{article}

%%%%%%%%%%%%%%%%%%%%%%%%%%%%%%%%%%%%%%%%%%%%%%%%%%%%%%%%%%%%%%%%%%%%%%%%%%%%%%%%
% packages
%%%%%%%%%%%%%%%%%%%%%%%%%%%%%%%%%%%%%%%%%%%%%%%%%%%%%%%%%%%%%%%%%%%%%%%%%%%%%%%%

\usepackage{coling2020}
\usepackage{times}
\usepackage{url}
\usepackage{latexsym}
\usepackage{hyperref}
\hypersetup{
  colorlinks   = true, %Colours links instead of ugly boxes
  urlcolor     = blue, %Colour for external hyperlinks
  linkcolor    = blue, %Colour of internal links
  citecolor    = blue  %Colour of citations
}


%%%%%%%%%%%%%%%%%%%%%%%%%%%%%%%%%%%%%%%%%%%%%%%%%%%%%%%%%%%%%%%%%%%%%%%%%%%%%%%%
% paper configuration
%%%%%%%%%%%%%%%%%%%%%%%%%%%%%%%%%%%%%%%%%%%%%%%%%%%%%%%%%%%%%%%%%%%%%%%%%%%%%%%%

%\setlength\titlebox{5cm}
%\colingfinalcopy % Uncomment this line for the final submission

% You can expand the titlebox if you need extra space
% to show all the authors. Please do not make the titlebox
% smaller than 5cm (the original size); we will check this
% in the camera-ready version and ask you to change it back.


\title{Instructions for COLING-2020 Proceedings}

\author{First Author \\
  Affiliation / Address line 1 \\
  Affiliation / Address line 2 \\
  Affiliation / Address line 3 \\
  {\tt email@domain} \\\And
  Second Author \\
  Affiliation / Address line 1 \\
  Affiliation / Address line 2 \\
  Affiliation / Address line 3 \\
  {\tt email@domain} \\}

\date{}

%%%%%%%%%%%%%%%%%%%%%%%%%%%%%%%%%%%%%%%%%%%%%%%%%%%%%%%%%%%%%%%%%%%%%%%%%%%%%%%%
% latex functions
%%%%%%%%%%%%%%%%%%%%%%%%%%%%%%%%%%%%%%%%%%%%%%%%%%%%%%%%%%%%%%%%%%%%%%%%%%%%%%%%


%%%%%%%%%%%%%%%%%%%%%%%%%%%%%%%%%%%%%%%%%%%%%%%%%%%%%%%%%%%%%%%%%%%%%%%%%%%%%%%%
% document text
%%%%%%%%%%%%%%%%%%%%%%%%%%%%%%%%%%%%%%%%%%%%%%%%%%%%%%%%%%%%%%%%%%%%%%%%%%%%%%%%

\begin{document}
\maketitle
\begin{abstract}
\end{abstract}

%
% The following footnote without marker is needed for the camera-ready
% version of the paper.
% Comment out the instructions (first text) and uncomment the 8 lines
% under "final paper" for your variant of English.
% 
\blfootnote{
    %
    % for review submission
    %
    \hspace{-0.65cm}  % space normally used by the marker
    Place licence statement here for the camera-ready version. 
    %
    % % final paper: en-uk version 
    %
    % \hspace{-0.65cm}  % space normally used by the marker
    % This work is licensed under a Creative Commons 
    % Attribution 4.0 International Licence.
    % Licence details:
    % \url{http://creativecommons.org/licenses/by/4.0/}.
    % 
    % % final paper: en-us version 
    %
    % \hspace{-0.65cm}  % space normally used by the marker
    % This work is licensed under a Creative Commons 
    % Attribution 4.0 International License.
    % License details:
    % \url{http://creativecommons.org/licenses/by/4.0/}.
}

\section{Introduction}
\label{sec:intro}

\cite{hu2005title} is the first work to consider title extraction.

\cite{gupta2003dom} is the classic work on using the DOM to extract the text body from the webpage.
\cite{weninger2010cetr,sun2011dom} also use DOM+heuristics.

Newer methods use visual representations of the webpage.
For example, \cite{DBLP:conf/paclic/Nguyen-HoangPJL18} renders the webpage and then processes using a CNN.
This ends up being more accurate, but too slow for large scale use.
IDEA: Try this on a small number of pages from each hostname to develop rules.
\cite{zeleny2017box,liu2017deep,song2015hybrid} also uses vision techniques.
\cite{gogar2016deep} also use deep neural networks.
Common to all these methods is that they are focusing on non-article pages.

\cite{wu2013web,carey2016html,wu2016language,wu2012extracting,DBLP:conf/widm/MohammadzadehGSH12,DBLP:journals/mms/UtomoL20,kim2013main,DBLP:conf/ic3k/MohammadzadehGSN11,endredy2013more} specifically focus on news.

\cite{alarte2014automatic,alarte2015site,alarte2015analysis,wu2016web} extract site-level templates,
which could potentially be used to make article extraction more accurate.

News-please \cite{Hamborg2017} is a similar system for downloading/extracting news content.
Had I known about it before starting this project, I probably would have used it as a base instead of newspaper3k.
Describes a heuristic result for comparing their extractor to other similar extractors.
See: https://github.com/fhamborg/news-please
Most similar work is python based, but \cite{khalil2017rcrawler} is an example in R.

\cite{ferrara2014web} older survey paper on lots of techniques.
\cite{schulz2016practical,schulz2016practical} surveys newer techniques.
\cite{berti2015veracity} is a book on truth detection.
\cite{aceto2015internet} is a survey on censorship detection.

Lots of research has a narrow focus.
\cite{qiu2015dexter} specifically focus on extracting product specifications.
\cite{catanese2011crawling} focus on crawling facebook (but it's old and the API changes regularly).
\cite{bakaev2014data} focuses on applications to monitoring the labor market.
\cite{audeh2017vigi4med} focus on forum posts.
\cite{nguyen2017design} focus on agricultural data.
\cite{kraychev2012computationally} focus on online reviews.
\cite{shi2015autorm} is a generic technique for domain-specific extractors.

\cite{jimenez2016ariex} provides a methodology for comparing the performance of different page extractors.

\bibliographystyle{coling}
\bibliography{main}

\end{document}

